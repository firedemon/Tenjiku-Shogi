\documentclass{article}
\usepackage{colordvi}
\usepackage{shogi}
\parindent0pt
\begin{document}

First, the three graphics from your starter kit:

\input KCPK1

\input fig4i

\input ideali

\bigskip
And now the \texttt{fig4i} with complete kanji:

\input fig4ifull

Finally, the Quick Attack part:

\subsection*{The Classical Quick Attack}

First we will look at White's attack on the eighth file, as
   exemplified by the historical game Mori versus Fukui (Game 1 in MSM, p.~11).

I present the opening of the game here, trancribed so that White is
playing down the board, in accordance with the modern convention.

1.\Fu--8h \Fu--5e 2.\Shi--6h \Fu--8e 3.\Fu--5h \Ho--5d 4.\Ho--8i \Fu--6e 5.\Fu--7h \Ki--6d 6.\Ki--7i \Ryu--6b 7.\Ryu7k \Fu--10e 8.\Hon--6k \Ma--11e 9.\Fu--3h \Ryu8c--7b 10.\Ma--2h \Hon--8c 11.\Fu--12h \Do--9b 12.\Do9--k \Do--9c 13.\Do--8j \Do--8d 14.\Shi--6g

\input fig1

So far, White has refrained from bringing out his Lion. This is to
leave the square 7e free for possible use by the Kylin. Therefore
Black has not bothered to take up the High Lion position, as it would
not immediately threaten anything (White has been controlling the
central squares on the f rank with Pawns and pieces). Now he finally
does so.

Both sides are playing a similar system. Black has brought up his
left-side Copper General before the right-side one, unlike White. This
suggests an intention to build a strong centre. On the other hand,
White has brought his right-side Copper up to support the impending
attack on the eighth file. This will enable him to retreat the Free
King quickly, then move the rook over to the eighth file---essential for an early attack.
  
Each side has brought out the right-side Dragon Horse, which supports
an attack on the head of the Phoenix. In contrast, bringing out the
left-side Dragon Horse would be a purely defensive measure at this
stage.

When the Kylin is jumped forward for an attack in the centre, it is
natural to place the Dragon Kings on 6b and 7b. Black has moved the
Free King instead, which is not quite so good, as the Dragon King
finds it harder to reach the back rank (also, it is desirable to have
the weaker piece in the centre, where the fighting is going to take
place. Still, Black's position is not bad.

14\ldots\Hon--10a 15.\Do4k \Ryu--8c 16.\Do--4j \Do--4b 17.\Do--5i \Ki--7e 18.\Fu--5g \Fu--6f 19.\Shi x!--6f \Shi--6e 20.\Shi--4g \Fu--3e 21.\Do--5h \Fu--8f 22.\Ki--6h \Do--8e 23.\Do--7i \Do--5c

\input fig2

White carries out his plan of \Fu, \Do{} and \Hi{} on the eighth
file, with close support from the Kylin. But first he has to sacrifice
a \Fu{} to free his position. Black correctly captured the offered
\Fu{}, then immediately moved his \Shi{} off the open file. He moved it
4g so as to leave 6h free for the \Ki, and 5h for the \Do, but
note that \Ma--2e will drive it away. I am not quite convinced of
Black's strategy. For my own preference, see ??.

Note that White could have started the middle game at move
twenty-three, with \Fu--8g, but he preferred to bring up more support.
This is generally good policy, but note that if instead of the good
shape of a Copper on 7i, Black had a Lion instead, then \Do--7f followed
by \Fu--8g (with \Ki--8f, then \Fu x8h, \Ki x8h, \Ki x8h, \Ho x8h, \Do--8g to
come) would succeed immediately.

24.\Ko--12i \Ken--10c 25.\Fu--12g \Fu--12e 26.\Chu--9g \Shi--7f 27.\Ho--10g \Ko--12d 28.\Chu--9h \Ma--2e 29.\Shi--5i \Fu--8g 30.\Ho--8i \Do--8f 31.\Ko--12h \Hyo--11b 32.\Gin--9k \Gin--9b 33.\Hyo--11k \Gin--9c

\input fig4i

Black starts activating his left flank with \Ko--12i. Although this is a
decent move, I would prefer to start with \Fu--10h and \Ma--11h, so as to
be able to play \Hi--8j.

White's response is to move his Vertical Mover across. As the Dragon
Horse will want to stay on 2e, this is the right spot for the Vertical
Mover, and it enlarges the attacking front, but I would prefer to
concentrate on the central attack, with moves like \Do--6d and \Fu--8g.

Black's next move reveals his defensive plan, and is definitely poor.
The idea of bringing the Side Mover to 12h to defend along the rank
may hold off the attack in the short term, but it will just make the
Side Mover a target for attack by White's right-side Ferocious
Leopard, followed by the Lion.

Black then makes a very poor attempt with his Go-Between and Phoenix,
which White refutes in excellent style, culminating in the attack with
\Fu--8g. Black is forced to retreat the Phoenix voluntarily (if he had
induced \Fu--10f from White, blocking the action of the Dragon Horse, he
might have felt justified, though I wouldn't take that attitude).
White is then content to bring up reinforcements. His position is much
to be preferred.



34.\Gin--8j \Gin--8d 35.\Ma--9k \Gin--8e 36.\Fu--1h \Fu--1e 37.\Fu--1g \Hyo--2b 38.\Ko--1i \Hyo--11c 39.\Chu--4g \Hyo--10d 40.\Fu--4h \Fu--10f 41.\Hyo--2k \Koii--9b 42.\Gin--4k \Hyo--10e 43.\Gin--4j \Hyo--11f 44.\Ko--11h \Kaku--9a 45.\Gin--4i \Shi--9f 46.\Koii--4k \Hyo--11g
   
And White picks up a material advantage, without losing his central
dominance. Later, White made a premature Lion sacrifice, and so
managed to lose the game.



\end{document}